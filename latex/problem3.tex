From the lectures we arrived at the following,

\[
  \gamma
  =
  1 + 
  \frac{1}{
    1 - \rho
  }
  \addtag
\]

For $\rho \to 0$,
\[
  \gamma
  =
  \lim_{\rho \to 0}
  1 + 
  \frac{1}{
    1 - \rho
  }
  =
  2
  \addtag
\]

For $\rho \to 1$,
\[
  \gamma
  =
  \lim_{\rho \to 1}
  1 + 
  \frac{1}{
    1 - \rho
  }
  =
  \infty
  \addtag
\]

For the case of $\gamma \to 2$, there is no mutation rate and the entire population is in the same grounp, this would cause the slope of the distribution to become infinite because there would be nothing to scale against. In the case where $\gamma \to \infty$, we are guarenteed to make a new group every time. This would cause the frequency of all the groups to be one, causing the power law slope to go to zero because each group occurs as often as every other one.